\documentclass[12pt,a4paper]{article}

\usepackage[T1]{fontenc}
\usepackage[polish]{babel}
\usepackage[utf8]{inputenc}
\usepackage{lmodern}
\selectlanguage{polish}
\usepackage{graphicx}

\begin{document}
\pagenumbering{gobble}
\clearpage
\begin{figure}[h]
\centering
\includegraphics{media/ps-logo.png}
\end{figure}
\hspace{3cm}
\begin{center}Programowanie obiektowe i graficzne\end{center}
\begin{center}2022/2023\end{center}
\hspace{3cm}
\begin{center}\large\textbf{Programowanie obiektowe i graficzne}\end{center}
\begin{center}\large\textit{Architektura MVP
}\end{center}

\hspace{7cm}
\begin{flushright}Kierunek: Informatyka
\end{flushright}
\begin{flushright}Członkowie zespołu:
\par
\textit{Jessica Adamczyk, grupa 1/1}
\par
\textit{Kateryna Dryzhakova, grupa 1/2}
\par
\textit{Żaneta Hofman, grupa 1/1}
\end{flushright}
\vfill
\begin{center}Gliwice, 2022/2023\end{center}

\newpage
\pagenumbering{arabic}
\tableofcontents

\newpage
\section{Wprowadzenie}

\subsection{Opis programu}
W ramach projektu została stworzona aplikacja do organizowania książek, które chce się przeczytać lub zostały już przeczytane. Aplikacja podzielona jest na dwie części – informacje o książce oraz ocenie. W części drugiej jest możliwe dodawanie własnej oceny książki oraz komentarz do niej.

\subsection{Instukcja obsługi}
Po rozpoczęciu działania programu uruchomiane zostaje okno aplikacji.
\begin{figure}[!h]
    \centering
    \includegraphics{media/2.png}
    \caption{Strona główna aplikacji}
\end{figure}
\clearpage
W oknie po lewej stronie wyświetla się pole do wyszukiwania tytułów, po wyszukaniu książki wyświetla się jej autor, wydawnictwo, opis oraz możliwość do zaznaczenia czy książka jest przeczytana czy w liście życzeń. 
\begin{figure}[!h]
    \centering
    \includegraphics{media/9.png}
    \caption{Wyszukana pozycja}
\end{figure}
\\
W przypadku braku pozycji w bazie wyskakuje komunikat.
\begin{figure}[!h]
    \centering
    \includegraphics{media/8.png}
    \caption{Komunikat o braku książki}
\end{figure}
\clearpage
Pod Moja Biblioteka jest wyświetlana ocena książki oraz przycisk gdzie po naciśnięciu otwiera się kolejne okno z komentarzami użytkowników do książki, którą wcześniej wyszukaliśmy. Pod tym jest możliwość dodania własnej opinii oraz komentarza.
\begin{figure}[!h]
    \centering
    \includegraphics{media/11.png}
    \caption{Opinie użytkowników}
\end{figure}
\clearpage
Po wybraniu opcji Moja Biblioteka otwiera się kolejne okno, w którym mamy dwie tabele, w pierwszej wypisane książki przeczytane, a w drugiej które chcemy przeczytać.
\begin{figure}[!h]
    \centering
    \includegraphics{media/7.png}
    \caption{Moja biblioteka}
\end{figure}

\subsection{Analiza i specyfikacja wymagań}
\begin{enumerate}
    \item Zaprojektowanie wyglądu aplikacji z łatwym i wygodnym w obsłudze interfejsem, aby umożliwić użytkownikom łatwe wyszukiwanie książek.
    \item Utworzenie bazy danych, która zawiera wiele książek. Każda pozycja ma tytuł, autora, krótki opis i opis autora.
    \item Dodanie funkcjonalności dodawania opinii i komentarzy przez użytkowników. To umożliwia innym użytkownikom podzielenie się swoimi opiniami na temat danej książki i pomaga w wyborze odpowiedniej lektury.
    \item Możliwość dodawania książek do listy życzeń lub książek przeczytanych, które pozwalają użytkownikom na kontrolowanie swojego postępu w czytaniu oraz na planowanie przyszłych lektur.
    \item Zaimportowanie bazy danych z książkami, ocenami i opisami. To pozwala na łatwe i szybkie zarządzanie książkami w aplikacji.
    \item Zaimportowanie bazy danych z książkami, ocenami i opisami. To pozwala na łatwe i szybkie zarządzanie książkami w aplikacji.
\end{enumerate}
\newpage

\section{Realizacja zadania}
\subsection{Opis działania}
Projekt jest wykonany w architekturze MVP, przy wykorzystaniu Forms. Implementacja aplikacji napisana została w języku programowania C\#, oraz współpracuje z serwerem MySQL. Aplikacja szuka danych w bazach danych i pozwala na zapis danych o stanie czytania naszej książki w osobnej bazie.
\subsection{Elementy aplikacji}
Aplikacja składa się z 3 okien (Strona główna, Moja biblioteka, Recenzje) oraz 3 istotnych klas, BookPresenter, MyLibraryPresenter oraz ReviewPresenter, zajmujących się osobnymi sferami obsługi danych - zarządzanie biblioteką i recenzjami oraz moją biblioteką.\\
\begin{figure}[!h]
    \centering
    \includegraphics{media/1.png}
    \caption{Struktura projektu}
\end{figure}
\clearpage
Model MVP polega na wykorzystywaniu View jako pośrednika pomiędzy Model a Presenter. W View znajduje się widok aplikacji, w Model deklaracja klas oraz pobieranie i ustawianie danych a w Presenter połączenie Model i View.
Połączenie wszystkich tych elementów, wsparte rozbudowanym systemem obsługi wyjątków sprawia, że projekt mimo swojego rozbudowania staje się jednocześnie przejrzysty i funkcjonalny.


W folderze Repository znajduje się odczytywanie bazy danych oraz zapis. 
\begin{figure}[!h]
    \centering
    \includegraphics{media/3.png}
    \caption{Repozytorium baz danych}
\end{figure}
\begin{figure}[!h]
    \centering
    \includegraphics{media/4.png}
    \caption{Widok Modelu}
\end{figure}
\begin{figure}[!h]
    \centering
    \includegraphics{media/5.png}
    \caption{Presenter}}
\end{figure}
\begin{figure}[!h]
    \centering
    \includegraphics{media/6.png}
    \caption{Odczyt danych}
\end{figure}
\clearpage
\subsection{Elementy bazy danych}
W programie zostały użyte dwie bazy danych, pierwsza to książki w której były stworzone następujące tabele: autorzy, książki, recenzje i wydawnictwo. Kolejna baza moja biblioteka w której zapisują się tytuły książek w dwóch tabelach: przeczytane i chce przeczytać. 
\section{Wnioski}
W programie zostały użyte dwie bazy danych, pierwsza to książki w której były stworzone następujące tabele: autorzy, książki, recenzje i wydawnictwo. Kolejna baza moja biblioteka w której zapisują się tytuły książek w dwóch tabelach: przeczytane i chce przeczytać. 

\end{document}

struktura projekt 
podzielone na warstwy
co gdzie jest jaki dostep do bazy danych
